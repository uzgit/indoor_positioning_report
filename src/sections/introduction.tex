This is a \LaTeX template document.\cite{small}

The DecaWave positioning system in V207 offers a solution for ultrawide band radio-based positioning,
via round-trip time of arrival distance estimation from a given tag to several anchors whose positions
are known.
It provides position estimation in 3 dimensions, but with low precision.
Initial usage of the system to play the video game Pong revealed difficulty in maintaining a consistent
position, which is problematic when trying to block the Pong ball from hitting the wall.
this project attempts to improve the DecaWave system's precision by using
multiple filtering methods: a Kalman filter, a particle filter, and a heading-aware filter that
assumes the vehicle whose position is being estimated can only move forward/backward
and rotate in the yaw dimension,
and attempts to ignore perceived side-to-side movement.
The data comes from 3 test cases, wherein a drone with one of the DecaWave tags, an IMU, a depth camera,
and an optical flow sensor contribute to its positioning accuracy.
Each of the proposed systems is a dead-reckoning system, meaning that it is subject to drift,
and therefore depends on the DecaWave system's ``global'' nature to consistently remove the drift.
The systems are designed purely to augment the DecaWave system, not to serve as a positioning
system on their own.
